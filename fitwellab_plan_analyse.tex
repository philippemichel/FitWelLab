% Options for packages loaded elsewhere
\PassOptionsToPackage{unicode}{hyperref}
\PassOptionsToPackage{hyphens}{url}
\PassOptionsToPackage{dvipsnames,svgnames,x11names}{xcolor}
%
\documentclass[
  a4paper,
  french,
  fontsize=10pt,
  oneside]{scrartcl}

\usepackage{amsmath,amssymb}
\usepackage{iftex}
\ifPDFTeX
  \usepackage[T1]{fontenc}
  \usepackage[utf8]{inputenc}
  \usepackage{textcomp} % provide euro and other symbols
\else % if luatex or xetex
  \usepackage{unicode-math}
  \defaultfontfeatures{Scale=MatchLowercase}
  \defaultfontfeatures[\rmfamily]{Ligatures=TeX,Scale=1}
\fi
\usepackage{lmodern}
\ifPDFTeX\else
    % xetex/luatex font selection
  \setmainfont[Ligatures=TeX]{Fira-sans}
  \setmonofont[]{Fira Mono}
\fi
% Use upquote if available, for straight quotes in verbatim environments
\IfFileExists{upquote.sty}{\usepackage{upquote}}{}
\IfFileExists{microtype.sty}{% use microtype if available
  \usepackage[]{microtype}
  \UseMicrotypeSet[protrusion]{basicmath} % disable protrusion for tt fonts
}{}
\makeatletter
\@ifundefined{KOMAClassName}{% if non-KOMA class
  \IfFileExists{parskip.sty}{%
    \usepackage{parskip}
  }{% else
    \setlength{\parindent}{0pt}
    \setlength{\parskip}{6pt plus 2pt minus 1pt}}
}{% if KOMA class
  \KOMAoptions{parskip=half}}
\makeatother
\usepackage{xcolor}
\usepackage[left=1in,marginparwidth=2.0in,textwidth=4.0in,marginparsep=0.3in]{geometry}
\setlength{\emergencystretch}{3em} % prevent overfull lines
\setcounter{secnumdepth}{5}
% Make \paragraph and \subparagraph free-standing
\ifx\paragraph\undefined\else
  \let\oldparagraph\paragraph
  \renewcommand{\paragraph}[1]{\oldparagraph{#1}\mbox{}}
\fi
\ifx\subparagraph\undefined\else
  \let\oldsubparagraph\subparagraph
  \renewcommand{\subparagraph}[1]{\oldsubparagraph{#1}\mbox{}}
\fi

\providecommand{\tightlist}{%
  \setlength{\itemsep}{0pt}\setlength{\parskip}{0pt}}\usepackage{longtable,booktabs,array}
\usepackage{calc} % for calculating minipage widths
% Correct order of tables after \paragraph or \subparagraph
\usepackage{etoolbox}
\makeatletter
\patchcmd\longtable{\par}{\if@noskipsec\mbox{}\fi\par}{}{}
\makeatother
% Allow footnotes in longtable head/foot
\IfFileExists{footnotehyper.sty}{\usepackage{footnotehyper}}{\usepackage{footnote}}
\makesavenoteenv{longtable}
\usepackage{graphicx}
\makeatletter
\def\maxwidth{\ifdim\Gin@nat@width>\linewidth\linewidth\else\Gin@nat@width\fi}
\def\maxheight{\ifdim\Gin@nat@height>\textheight\textheight\else\Gin@nat@height\fi}
\makeatother
% Scale images if necessary, so that they will not overflow the page
% margins by default, and it is still possible to overwrite the defaults
% using explicit options in \includegraphics[width, height, ...]{}
\setkeys{Gin}{width=\maxwidth,height=\maxheight,keepaspectratio}
% Set default figure placement to htbp
\makeatletter
\def\fps@figure{htbp}
\makeatother

\usepackage{booktabs}
\usepackage{longtable}
\usepackage{array}
\usepackage{multirow}
\usepackage{wrapfig}
\usepackage{float}
\usepackage{colortbl}
\usepackage{pdflscape}
\usepackage{tabu}
\usepackage{threeparttable}
\usepackage{threeparttablex}
\usepackage[normalem]{ulem}
\usepackage{makecell}
\usepackage{xcolor}
\definecolor{novo}{HTML}{27484b}
\usepackage{siunitx}
\AddToHook{env/tabular/before}{\addfontfeatures{Numbers=Monospaced}}
\AddToHook{env/longtable/before}{\addfontfeatures{Numbers=Monospaced}}
\usepackage{alphabeta}
\usepackage{marginnote}
\renewcommand*{\marginfont}{\footnotesize}
\makeatletter
\@ifpackageloaded{tcolorbox}{}{\usepackage[skins,breakable]{tcolorbox}}
\@ifpackageloaded{fontawesome5}{}{\usepackage{fontawesome5}}
\definecolor{quarto-callout-color}{HTML}{909090}
\definecolor{quarto-callout-note-color}{HTML}{0758E5}
\definecolor{quarto-callout-important-color}{HTML}{CC1914}
\definecolor{quarto-callout-warning-color}{HTML}{EB9113}
\definecolor{quarto-callout-tip-color}{HTML}{00A047}
\definecolor{quarto-callout-caution-color}{HTML}{FC5300}
\definecolor{quarto-callout-color-frame}{HTML}{acacac}
\definecolor{quarto-callout-note-color-frame}{HTML}{4582ec}
\definecolor{quarto-callout-important-color-frame}{HTML}{d9534f}
\definecolor{quarto-callout-warning-color-frame}{HTML}{f0ad4e}
\definecolor{quarto-callout-tip-color-frame}{HTML}{02b875}
\definecolor{quarto-callout-caution-color-frame}{HTML}{fd7e14}
\makeatother
\makeatletter
\makeatother
\makeatletter
\makeatother
\makeatletter
\@ifpackageloaded{caption}{}{\usepackage{caption}}
\AtBeginDocument{%
\ifdefined\contentsname
  \renewcommand*\contentsname{Table des matières}
\else
  \newcommand\contentsname{Table des matières}
\fi
\ifdefined\listfigurename
  \renewcommand*\listfigurename{Liste des Figures}
\else
  \newcommand\listfigurename{Liste des Figures}
\fi
\ifdefined\listtablename
  \renewcommand*\listtablename{Liste des Tables}
\else
  \newcommand\listtablename{Liste des Tables}
\fi
\ifdefined\figurename
  \renewcommand*\figurename{Figure}
\else
  \newcommand\figurename{Figure}
\fi
\ifdefined\tablename
  \renewcommand*\tablename{Table}
\else
  \newcommand\tablename{Table}
\fi
}
\@ifpackageloaded{float}{}{\usepackage{float}}
\floatstyle{ruled}
\@ifundefined{c@chapter}{\newfloat{codelisting}{h}{lop}}{\newfloat{codelisting}{h}{lop}[chapter]}
\floatname{codelisting}{Listing}
\newcommand*\listoflistings{\listof{codelisting}{Liste des Listings}}
\makeatother
\makeatletter
\@ifpackageloaded{caption}{}{\usepackage{caption}}
\@ifpackageloaded{subcaption}{}{\usepackage{subcaption}}
\makeatother
\makeatletter
\@ifpackageloaded{tcolorbox}{}{\usepackage[skins,breakable]{tcolorbox}}
\makeatother
\makeatletter
\@ifundefined{shadecolor}{\definecolor{shadecolor}{rgb}{.97, .97, .97}}
\makeatother
\makeatletter
\makeatother
\makeatletter
\@ifpackageloaded{sidenotes}{}{\usepackage{sidenotes}}
\@ifpackageloaded{marginnote}{}{\usepackage{marginnote}}
\makeatother
\makeatletter
\@ifpackageloaded{biblatex}{}{\usepackage{biblatex}}
\makeatother
\makeatletter
\makeatother

\usepackage{hyphenat}
\usepackage{ifthen}
\usepackage{calc}
\usepackage{calculator}

\usepackage{graphicx}
\usepackage{wallpaper}

\usepackage{geometry}

\usepackage{graphicx}
\usepackage{geometry}
\usepackage{afterpage}
\usepackage{tikz}
\usetikzlibrary{calc}
\usetikzlibrary{fadings}
\usepackage[pagecolor=none]{pagecolor}


% Set the titlepage font families







% Set the coverpage font families

\ifLuaTeX
\usepackage[bidi=basic]{babel}
\else
\usepackage[bidi=default]{babel}
\fi
\babelprovide[main,import]{french}
% get rid of language-specific shorthands (see #6817):
\let\LanguageShortHands\languageshorthands
\def\languageshorthands#1{}
\ifLuaTeX
  \usepackage{selnolig}  % disable illegal ligatures
\fi
\usepackage[]{biblatex}
\addbibresource{stat.bib}
\IfFileExists{bookmark.sty}{\usepackage{bookmark}}{\usepackage{hyperref}}
\IfFileExists{xurl.sty}{\usepackage{xurl}}{} % add URL line breaks if available
\urlstyle{same} % disable monospaced font for URLs
\hypersetup{
  pdftitle={FitWelLAB},
  pdfauthor={D Philippe },
  pdflang={fr},
  colorlinks=true,
  linkcolor={blue},
  filecolor={Maroon},
  citecolor={Blue},
  urlcolor={Blue},
  pdfcreator={LaTeX via pandoc}}

\title{FitWelLAB}
\usepackage{etoolbox}
\makeatletter
\providecommand{\subtitle}[1]{% add subtitle to \maketitle
  \apptocmd{\@title}{\par {\large #1 \par}}{}{}
}
\makeatother
\subtitle{Plan d'analyse statistique}
\author{D\textsuperscript{r} Philippe \textsc{Michel}}
\date{}

\begin{document}
%%%%% begin titlepage extension code


\begin{titlepage}

%%% TITLE PAGE START

% Set up alignment commands
%Page
\newcommand{\titlepagepagealign}{
\ifthenelse{\equal{left}{right}}{\raggedleft}{}
\ifthenelse{\equal{left}{center}}{\centering}{}
\ifthenelse{\equal{left}{left}}{\raggedright}{}
}


\newcommand{\titleandsubtitle}{
% Title and subtitle
{\textcolor{novo}{\Huge{\bfseries{\nohyphens{FitWelLAB}}}}\par
}%

\vspace{\betweentitlesubtitle}
{
\textcolor{novo}{\huge{\nohyphens{Plan d'analyse statistique}}}\par
}}
\newcommand{\titlepagetitleblock}{
\titleandsubtitle
}

\newcommand{\authorstyle}[1]{{\large{#1}}}

\newcommand{\affiliationstyle}[1]{{\large{#1}}}

\newcommand{\titlepageauthorblock}{
{\authorstyle{\nohyphens{D\textsuperscript{r} Philippe
\textsc{Michel}}{\textsuperscript{1}}}}}

\newcommand{\titlepageaffiliationblock}{
\hangindent=1em
\hangafter=1
{\affiliationstyle{
{1}.~Hôpital \textsc{novo},~Unité de Soutien à la Recherche Clinique


\vspace{1\baselineskip}
}}
}
\newcommand{\headerstyled}{%
{}
}
\newcommand{\footerstyled}{%
{\large{Djeneba \textsc{Camara} \newline Hôpital \textsc{novo} (Site
Pontoise)\newline \newline Chef de projet : M\up{me} Mathilde
\textsc{Wlodarczyk}\newline \newline Impact de la section associative et
sportive Section Athlé \textsc{novo} sur le bien-être au travail des
agents de l'hôpital \textsc{novo} \newline \newline \today}}
}
\newcommand{\datestyled}{%
{}
}


\newcommand{\titlepageheaderblock}{\headerstyled}

\newcommand{\titlepagefooterblock}{
\footerstyled
}

\newcommand{\titlepagedateblock}{
\datestyled
}

%set up blocks so user can specify order
\newcommand{\titleblock}{\newlength{\betweentitlesubtitle}
\setlength{\betweentitlesubtitle}{\baselineskip}
{

{\titlepagetitleblock}
}

\vspace{4\baselineskip}
}

\newcommand{\authorblock}{{\titlepageauthorblock}

\vspace{2\baselineskip}
}

\newcommand{\affiliationblock}{{\titlepageaffiliationblock}

\vspace{1pt}
}

\newcommand{\logoblock}{}

\newcommand{\footerblock}{{\titlepagefooterblock}

\vspace{1pt}
}

\newcommand{\dateblock}{}

\newcommand{\headerblock}{}
\newgeometry{top=3in,bottom=1in,right=1in,left=1in}
% background image
\newlength{\bgimagesize}
\setlength{\bgimagesize}{0.5\paperwidth}
\LENGTHDIVIDE{\bgimagesize}{\paperwidth}{\theRatio} % from calculator pkg
\ThisULCornerWallPaper{\theRatio}{novo\_usrc.png}

\thispagestyle{empty} % no page numbers on titlepages


\newcommand{\vrulecode}{\rule{\vrulewidth}{\textheight}}
\newlength{\vrulewidth}
\setlength{\vrulewidth}{0.1cm}
\newlength{\B}
\setlength{\B}{\ifdim\vrulewidth > 0pt 0.05\textwidth\else 0pt\fi}
\newlength{\minipagewidth}
\ifthenelse{\equal{left}{left} \OR \equal{left}{right} }
{% True case
\setlength{\minipagewidth}{\textwidth - \vrulewidth - \B - 0.1\textwidth}
}{
\setlength{\minipagewidth}{\textwidth - 2\vrulewidth - 2\B - 0.1\textwidth}
}
\ifthenelse{\equal{left}{left} \OR \equal{left}{leftright}}
{% True case
\raggedleft % needed for the minipage to work
\vrulecode
\hspace{\B}
}{%
\raggedright % else it is right only and width is not 0
}
% [position of box][box height][inner position]{width}
% [s] means stretch out vertically; assuming there is a vfill
\begin{minipage}[b][\textheight][s]{\minipagewidth}
\titlepagepagealign
\titleblock

\authorblock

\affiliationblock

\vfill

\logoblock

\footerblock
\par

\end{minipage}\ifthenelse{\equal{left}{right} \OR \equal{left}{leftright} }{
\hspace{\B}
\vrulecode}{}
\clearpage
\restoregeometry
%%% TITLE PAGE END
\end{titlepage}
\setcounter{page}{1}

%%%%% end titlepage extension code\ifdefined\Shaded\renewenvironment{Shaded}{\begin{tcolorbox}[enhanced, interior hidden, sharp corners, borderline west={3pt}{0pt}{shadecolor}, breakable, frame hidden, boxrule=0pt]}{\end{tcolorbox}}\fi

\renewcommand*\contentsname{Table des matières}
{
\hypersetup{linkcolor=}
\setcounter{tocdepth}{3}
\tableofcontents
}
\newpage

\hypertarget{guxe9nuxe9ralituxe9s}{%
\section{Généralités}\label{guxe9nuxe9ralituxe9s}}

Le risque \textalpha{} retenu sera de 0,05 \& la puissance de 0,8.

Vu le faible nombre de cas on ne fera pas d'hypothèse de normalité.
Toutes les variables sont discrètes \& seront présentés en nombre avec
le pourcentage. L'intervalle de confiance (à 95 \%) sera calculé par
bootstrap (package \texttt{boot}
\autocite{boot}\marginpar{\begin{footnotesize}\fullcite{boot}\vspace{2mm}\par\end{footnotesize}}).
Le test du \textchi\(^2\) de \textsc{Spearman} sera utilisé sous réserve
d'un effectif suffisant, à défaut le test exact de \textsc{Fischer}.

Des graphiques seront réalisés pour les résultats importants (package
\texttt{ggplot2}
\autocite{ggplot}\marginpar{\begin{footnotesize}\fullcite{ggplot}\vspace{2mm}\par\end{footnotesize}})
en particulier des graphiques en fagot pour montrer les évolutions pour
les items jugés importants (forte évolution, items récapitulatifs\dots).

Des graphiques de flux (package \texttt{network3D}
\autocite{reseau}\marginpar{\begin{footnotesize}\fullcite{reseau}\vspace{2mm}\par\end{footnotesize}})
pourront être réalisés au besoin.

\begin{tcolorbox}[enhanced jigsaw, colbacktitle=quarto-callout-warning-color!10!white, opacityback=0, toprule=.15mm, leftrule=.75mm, colback=white, toptitle=1mm, colframe=quarto-callout-warning-color-frame, titlerule=0mm, left=2mm, bottomtitle=1mm, title=\textcolor{quarto-callout-warning-color}{\faExclamationTriangle}\hspace{0.5em}{Avertissement}, coltitle=black, arc=.35mm, rightrule=.15mm, bottomrule=.15mm, breakable, opacitybacktitle=0.6]

Une analyse complète nécessiterait des tests multiples impossibles à
réaliser ici vu la taille de l'échantillon.

On se contentera de tableau purement descriptifs.

\end{tcolorbox}

\hypertarget{taille-de-luxe9chantillon}{%
\subsection{Taille de l'échantillon}\label{taille-de-luxe9chantillon}}

La taille de l'échantillon est limitée par le nombre de personnes
participant à ce programme \marginnote{15 à ce jour.}.

Si on considère qu'il s'agit d'une étude avant/après donc sur séries
appariées avec une différence jugée significative à 0,5 (écart-type
1)\marginnote{Chiffres très approximatifs en l'absence de données préalables}
il faudrait un strict minimum de 32 patients dans l'étude.

\hypertarget{donnuxe9es-manquantes-contruxf4le-qualituxe9}{%
\subsection{Données manquantes \& contrôle
qualité}\label{donnuxe9es-manquantes-contruxf4le-qualituxe9}}

Un décompte des données manquantes sera réalisé \& présenté par un
tableau ou un graphique. Les variables comportant trop de données
manquantes ou non utilisables ne seront pas prises en compte après
validation par le promoteur.

Vu le très faible effectif une validation interne des questionnaires
(coefficient \textalpha{} de \textsc{Cronbach}) ne pourra être réalisé.

\hypertarget{description-de-la-population}{%
\section{Description de la
population}\label{description-de-la-population}}

Un tableau \& des graphiques (pyramide des âges\dots) présenteront les
données démographiques c'est à dire la \texttt{Section\ 1} du
questionnaire \textsc{satin}.

\hypertarget{objectif-principal}{%
\section{Objectif principal}\label{objectif-principal}}

\textbf{Évaluation de l'impact de l'intervention (section Athlé NOVO)
sur le bien-être au travail.}

Quatre tableaux présenteront les réponses par section du questionnaire
\textsc{satin}\marginnote{%
- Santé perçue\\
- Efforts requis \& capacité disponible\\
- Caractéristiques de l'environnement de travail\\
- Appréciation générale} remplis à M0, M3, M7 \& complétés par des
graphiques.

\hypertarget{objectifs-secondaires}{%
\section{Objectifs secondaires}\label{objectifs-secondaires}}

\hypertarget{objectif-1}{%
\subsection{Objectif 1}\label{objectif-1}}

\textbf{Un bilan de l'activité physique des participants sera réalisé
grâce au remplissage du questionnaire IPAQ (version courte de 7
questions) à 0, 5 et 9 mois du début de leur participation à la
section.}

Un tableau \& des graphiques présentant les réponses aux 7 items du
questionnaire \textsc{ipaq} à M0,M3 \& M7 seront réalisés.

\hypertarget{objectif-2}{%
\subsection{Objectif 2}\label{objectif-2}}

\textbf{Les objectifs des adhérents seront recueillis grâce au
questionnaire des objectifs, au début de leur participation à la
section.}

Un tableau \& des graphiques présentant les réponses aux questionnaire
des objectifs à M0 seront réalisés.

\hypertarget{objectif-3}{%
\subsection{Objectif 3}\label{objectif-3}}

\textbf{La satisfaction des adhérents sera évaluée grâce au remplissage
d'un questionnaire à 3 et 7 mois du début de leur participation à la
section.}

Un tableau \& des graphiques (package \texttt{likert}
\autocite{lik}\marginpar{\begin{footnotesize}\fullcite{lik}\vspace{2mm}\par\end{footnotesize}})
présentant les réponses aux questionnaires de satisfaction (échelles de
Likert) à M3 \& M7 seront réalisés.

\hypertarget{technique}{%
\section{Technique}\label{technique}}

L'analyse statistique sera réalisée avec le logiciel
\textbf{R}\autocite{rstat}\marginpar{\begin{footnotesize}\fullcite{rstat}\vspace{2mm}\par\end{footnotesize}}
\& divers packages. Outre ceux cités dans le texte ou utilisera en
particulier \texttt{tidyverse}
\autocite{tidy}\marginpar{\begin{footnotesize}\fullcite{tidy}\vspace{2mm}\par\end{footnotesize}}
\& \texttt{baseph}
\autocite{baseph}\marginpar{\begin{footnotesize}\fullcite{baseph}\vspace{2mm}\par\end{footnotesize}}.

Un dépôt GitHub sera utilisé qui ne comprendra que le code \& non les
données ou résultats. Au besoin un faux tableau de données sera présenté
pour permettre des tests.

\url{https://github.com/philippemichel/fitwellab}





\end{document}
